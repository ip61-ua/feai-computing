\documentclass{article}

\usepackage[utf8]{inputenc}
\usepackage{amsmath}
\usepackage[left=2cm, right=2cm, top=2cm, bottom=2cm]{geometry}
\title{Exámenes de ADA}

\begin{document}
	\section{Propiedades de O (cota superior, peor caso)}
	{\LARGE
	\begin{equation}
		f \in O(f)
	\end{equation}
	\begin{equation}
		f \in O(g) \implies O(f) \subseteq O(g)
	\end{equation}
	\begin{equation}
		O(f) = O(g) \iff f \in O(g) \land g \in O(f)
	\end{equation}
	\begin{equation}
		f \in O(g) \land g \in O(h) \implies f \in O(h) 
	\end{equation}
	\begin{equation}
		f \in O(g) \land g \in O(h) \implies f \in O(\min \{g, h\})
	\end{equation}
	\begin{equation}
		f_1 \in O(g_1) \land f_2 \in O(g_2) \implies f_1 + f_2 \in O(g_1 + g_2) 
	\end{equation}
	\begin{equation}
		f_1 \in O(g_1) \land f_2 \in O(g_2) \implies f_1 + f_2 \in O(\max \{g_1, g_2\})
	\end{equation}
	\begin{equation}
		f_1 \in O(g_1) \land f_2 \in O(g_2) \implies f_1 f_2 \in O(g_1 g_2)
	\end{equation}	
	\begin{equation}
		f(n) = a_m n^m + ... + a_1 n + a_0 \text{ tal que } a_m > 0 \implies f \in O(n^m)
	\end{equation}	
	\begin{equation}
		O(f) \subset O(g) \implies f \in O(g) \land g \notin O(f)
	\end{equation}	
	}
	
	\paragraph{Ejercicio 1.1.} Resuelve.
	{\LARGE
	\begin{displaymath}
		f(n)=\log (n) + \log_2 (n^3) + \log_3 (n^2) + n \log(n) \in O(?)
	\end{displaymath}
	}
	
	Aplicamos
	\begin{displaymath}
		f_1 \in O(g_1) \land f_2 \in O(g_2) \implies f_1 + f_2 \in O(\max \{g_1, g_2\})
	\end{displaymath}
	\begin{displaymath}
		f(n) \in O(\max \{\log(n), \log_2(n^3), \log_3(n^2), n\log(n)\})
	\end{displaymath}
	\begin{displaymath}
		O(\max \{\log(n), \log_2(n^3), \log_3(n^2), n\log(n)\}) = O(\max \{\log(n), n \log(n)\}) = O(n \log(n))
	\end{displaymath}
	
	Por lo tanto, \[ f(n) \in O(nlog(n))\]
	{\LARGE
	\begin{displaymath}
		g(n) = 2^{\log_2 (n^2)} + 4^{\log_2 (n)} + 2^{\log_2 (n)} \in O(?)
	\end{displaymath}
	}
	
	Procedemos con
	\[
		f_1 \in O(g_1) \land f_2 \in O(g_2) \implies f_1 + f_2 \in O(\max \{g_1, g_2\})
	\]
	\[
		g(n) \in O(\max \{2^{\log_2 (n^2)}, 4^{\log_2 (n)}, 2^{\log_2 (n)}\})
	\]
	
	Podemos simplificar recurriendo a $a^{\log_a (x)} = x$. Obteniendo: $2^{\log_2 (n^2)} = n^2$. Así como $2^{\log_2(n)} = n$
	
	Simplificando $4^{\log_2 (n)} = 2^{2\log_2 (n)} = n^2$
	
	Quedando así: $g \in O(\max \{ n^2, n \})$
	
	Por lo tanto,
	\[
		g \in O(n^2)
	\]
	{\LARGE
	\begin{displaymath}
		f (n) + g(n) \in O(?)
	\end{displaymath}
	}

	Aplicación directa
	\[
		f + g \in O(n \log(n) + n^2) \qquad f + g \in O(n^2)
	\]
	
	\paragraph{Ejercicio 1.2.} ¿Verdadero o falso?
	{\LARGE
	\begin{displaymath}
		4n^3-2n^2+8 \in O(n^3)
	\end{displaymath}
	}
	\begin{displaymath}
		f_1 \in O(g_1) \land f_2 \in O(g_2) \implies f_1 + f_2 \in O(\max \{g_1, g_2\}) \qquad 4n^3-2n^2+8 \in O(\max \{4n^3, -2n^2, 8\})
	\end{displaymath}
	\begin{displaymath}
		4n^3-2n^2+8 \in O(\max \{n^3, n^2, 1\}) \qquad O(1) \subset O(n^2) \subset O(n^3) \qquad 4n^3-2n^2+8 \in O(n^3) \qquad \text{Verdadero.}
	\end{displaymath}
	{\LARGE
	\begin{displaymath}
		n^3 \in O(4n^3-2n^2+8)
	\end{displaymath}
	}
	\[
		f \in O(g) \land g \in O(h) \implies f \in O(h) \qquad 4n^3-2n^2+8 \in O(n^3) \qquad n^3 \in O(n^3) \qquad \text{Verdadero.}
	\]
	{\LARGE
	\begin{displaymath}
		n + n\sqrt{n} \in O(n)
	\end{displaymath}
	}
	\[
		f_1 \in O(g_1) \land f_2 \in O(g_2) \implies O(\max \{g_1,g_2\}) \qquad n + n\sqrt{n} \in O(\max \{n, n\sqrt{n}\}) \qquad O(n) \subset O(n\sqrt{n})
	\]
	\[
		n + n\sqrt{n} \in O(n\sqrt{n}) \qquad n\sqrt{n} \notin O(n) \qquad \text{Falso.}
	\]
	{\LARGE
	\begin{displaymath}
		n + n\sqrt{n} \in \Theta(n)
	\end{displaymath}
	}
	\[
		\text{Por el mismo motivo, es falso.}
	\]
	{\LARGE
	\begin{displaymath}
		f\notin O(g) \land \exists \lim\limits_{n \to +\infty}{\frac{f}{g}} \implies f \in \Omega (g)
	\end{displaymath}
	}
	\[
		\lim\limits_{n \to \infty}{\frac f g} = 0 \iff f\in O(g) \land g \notin O(f) \quad 
		\lim\limits_{n \to \infty}{\frac f g} = +\infty \iff f\notin O(g) \land g \in O(f) \quad 
		\lim\limits_{n \to \infty}{\frac f g} = \Re^+ \iff f\in O(g) \land g \in O(f)
	\]
	\[
		\lim\limits_{n \to \infty}{\frac f g} = 0 \iff f\notin \Omega(g) \land g \in \Omega(f) \quad 
		\lim\limits_{n \to \infty}{\frac f g} = +\infty \iff f\in \Omega(g) \land g \notin \Omega(f) \quad 
		\lim\limits_{n \to \infty}{\frac f g} = \Re^+ \iff f\in \Omega(g) \land g \in \Omega(f)
	\]
	\[
		f\notin O(g) \implies \lim\limits_{n \to +\infty}{\frac{f}{g}}=+\infty \implies f\in \Omega(g) \land g \notin \Omega(f) \qquad \text{Verdadero.}
	\]

\end{document}
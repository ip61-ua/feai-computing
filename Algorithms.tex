\documentclass{article}

\usepackage[utf8]{inputenc}
\usepackage{amsmath}
\usepackage[left=2cm, right=2cm, top=2cm, bottom=2cm]{geometry}
\title{Exámenes de ADA}

\begin{document}
	\section{Propiedades de O (cota superior, peor caso)}
	{\LARGE
	\begin{equation}
		f \in O(f)
	\end{equation}
	\begin{equation}
		f \in O(g) \implies O(f) \subseteq O(g)
	\end{equation}
	\begin{equation}
		O(f) = O(g) \iff f \in O(g) \land g \in O(f)
	\end{equation}
	\begin{equation}
		f \in O(g) \land g \in O(h) \implies f \in O(h) 
	\end{equation}
	\begin{equation}
		f \in O(g) \land g \in O(h) \implies f \in O(\min \{g, h\})
	\end{equation}
	\begin{equation}
		f_1 \in O(g_1) \land f_2 \in O(g_2) \implies f_1 + f_2 \in O(g_1 + g_2) 
	\end{equation}
	\begin{equation}
		f_1 \in O(g_1) \land f_2 \in O(g_2) \implies f_1 + f_2 \in O(\max \{g_1, g_2\})
	\end{equation}
	\begin{equation}
		f_1 \in O(g_1) \land f_2 \in O(g_2) \implies f_1 f_2 \in O(g_1 g_2)
	\end{equation}	
	\begin{equation}
		f(n) = a_m n^m + ... + a_1 n + a_0 \text{ tal que } a_m > 0 \implies f \in O(n^m)
	\end{equation}	
	\begin{equation}
		O(f) \subset O(g) \implies f \in O(g) \land g \notin O(f)
	\end{equation}	
	}
	
	\paragraph{Ejercicio 1.1.} Resuelve.
	{\LARGE
	\begin{displaymath}
		f(n)=\log (n) + \log_2 (n^3) + \log_3 (n^2) + n \log(n) \in O(?)
	\end{displaymath}
	}
	
	Aplicamos
	\begin{displaymath}
		f_1 \in O(g_1) \land f_2 \in O(g_2) \implies f_1 + f_2 \in O(\max \{g_1, g_2\})
	\end{displaymath}
	\begin{displaymath}
		f(n) \in O(\max \{\log(n), \log_2(n^3), \log_3(n^2), n\log(n)\})
	\end{displaymath}
	\begin{displaymath}
		O(\max \{\log(n), \log_2(n^3), \log_3(n^2), n\log(n)\}) = O(\max \{\log(n), n \log(n)\}) = O(n \log(n))
	\end{displaymath}
	
	Por lo tanto, \[ f(n) \in O(nlog(n))\]
	{\LARGE
	\begin{displaymath}
		g(n) = 2^{\log_2 (n^2)} + 4^{\log_2 (n)} + 2^{\log_2 (n)} \in O(?)
	\end{displaymath}
	}
	
	Procedemos con
	\[
		f_1 \in O(g_1) \land f_2 \in O(g_2) \implies f_1 + f_2 \in O(\max \{g_1, g_2\})
	\]
	\[
		g(n) \in O(\max \{2^{\log_2 (n^2)}, 4^{\log_2 (n)}, 2^{\log_2 (n)}\})
	\]
	
	Podemos simplificar recurriendo a $a^{\log_a (x)} = x$. Obteniendo: $2^{\log_2 (n^2)} = n^2$. Así como $2^{\log_2(n)} = n$
	
	Simplificando $4^{\log_2 (n)} = 2^{2\log_2 (n)} = n^2$
	
	Quedando así: $g \in O(\max \{ n^2, n \})$
	
	Por lo tanto,
	\[
		g \in O(n^2)
	\]
	{\LARGE
	\begin{displaymath}
		f (n) + g(n) \in O(?)
	\end{displaymath}
	}

	Aplicación directa
	\[
		f + g \in O(n \log(n) + n^2) \qquad f + g \in O(n^2)
	\]
\end{document}
\documentclass{article}
\usepackage[utf8]{inputenc}
\usepackage{multicol}
\usepackage{amsmath}
\usepackage{geometry}

\geometry{top=2cm, bottom=2cm, left=2cm, right=2cm}
\title{Arquitectura de los computadores}
\author{ip61@alu.ua.es}
\date{\today}


\begin{document}
	\maketitle
	
	\begin{multicols}{2}
	\section{Introducción}
		En la actualidad, es una disciplina dirigida al diseño de máquinas para ejecutar programas en base a la optimización entre coste, rendimiento y otras restricciones. Alguno de los ámbitos que comprende:
		\begin{itemize}
			\item Repertorio de instrucciones.
			\item Organización e implementación de la estructura del sistema de memoria, bus datos...
			\item Diseño lógico del \textit{hardware}.
		\end{itemize}
		
		\subsection{Definición de arquitectura.}

		El concepto arquitectura de un computador se define formalmente en 1964 por Amdahl en IBM como parte del repertorio de instrucciones visible por el programador.
		\paragraph{}
		Ortega en 2005 da una definición más extensa: "\textit{Conjunto de instrucciones, recursos y características del procesador que son visibles al software que se ejecuta en el mismo. Por tanto, la arquitectura determina el software que el procesador puede ejecutar ...}" 
		\paragraph{}
		Además, Ortega define la microarquitectura como "\textit{Conjunto de recursos y métodos utilizados para satisfacer las especificaciones que establece la arquitectura ... La microarquitectura define las especificaciones para la implementación lógica.}"
		

		\subsection{Niveles de abstracción.}
		Unificando los niveles de Bell junto a Newell y la visión funcional de Levy y Tanenbaum, permitiendo definir la jerarquía de abstracción por distintos niveles:
		 \newline
		 \newline
		\begin{tabular}{ll}
			Nivel. & Categoría. \\
			4, 5 y 6. & Arquitectura (\textit{Software}). \\
			3. & Entre ambas agrupaciones. \\
			0, 1 y 2. & Tecnología (\textit{Hardware}). \\
		\end{tabular}
		
		\columnbreak
		
		\paragraph{Nivel 6. Niveles superiores de \textit{software}.}
		Lenguajes de alto nivel. Compiladores. Interfaces de usuario... 
		\paragraph{Nivel 5. Sistema operativo.}
		Interfaz entre \textit{hardware} y \textit{software}. Administración de tareas.
		\paragraph{Nivel 4. Sistema computador.}
		Especificación, operación e interconexión entre componentes (buses, memorias, periféricos...).
		
		Programación en lenguaje máquina y ensamblador.
		\paragraph{Nivel 3. Transferencia entre registros.}
		Nivel destinado al estudio sobre la transferencia de información. Microprogramación. Procesadores, Interfaces de entrada y salida... 
		
		Es la división entre arquitectura y tecnología.
		\paragraph{Nivel 2. Digital.}
		A nivel combinacional: MUX, DEMUX, circuitos aritméticos... 
		A nivel secuencial: registros, contadores, memoria...
		\paragraph{Nivel 1. Electrónica.}
		Cimentado en las leyes de electricidad. Se construyen puertas lógicas, biestables... mediante corrientes, tensiones y frecuencias.
		\paragraph{Nivel 0. Componente.}
		Fundamentado en las leyes físicas. Son semiconductores N/P, metales y otros materiales físicos.
		
		\subsection{Taxonomía de Flynn.}
		
		\paragraph{} 
		S = Simple, M = multiple, I = instruction y D = Data.
		\paragraph{SISD}
		Una instrucción, un flujo de datos. Arquitectura Von Neumann. MIPS.
		\paragraph{SIMD}
		Una instrucción, muchos datos. Su uso está muy extendido en las GPUs y procesadores vectoriales. Tienen paralelismo.
		\paragraph{MISD} 
		No hay ejemplos reales.
		\paragraph{MIMD} 
		Muchas instrucciones operan sobre muchos datos. También presentan paralelismo.

		\subsection{Paralelismo funcional.}
		
		Es un tipo de paralelismo (ejecución en paralelo sobre distintos datos) aplicado a rutinas, como instrucciones o funciones. Distinguiendo entre:
		\paragraph{Instrucción (ILP).}
		Ejecución paralela sobre una serie de instrucciones. Granularidad fina.
		\paragraph{Bucle o hebra (\textit{Thread}).} 
		En paralelo, se ejecuta distintas iteraciones de un bucle. Granularidad fina/media.
		\paragraph{Procedimiento (Proceso).} 
		Simultaneidad en la ejecución de procedimientos en un programa. Granularidad media.
		\paragraph{Programa.} 
		Hay concurrencia entre los distintos programas/aplicaciones. Tienen una granularidad gruesa.
		
		\subsection{Tipos de computadores.}
		\paragraph{Dispositivos móviles personales.}
		Buscan equilibrio entre coste y consumo energético: teléfonos móviles, ordenadores portátiles (\textit{laptops}), videoconsolas portables...
		\paragraph{Ordenadores sobremesa.} 
		Existe un equilibrio precio-rendimiento según las necesidades: ofimática, multimedia, videojuegos,  gráficos...
		\paragraph{Servidores.} 
		Este tipo de dispositivos se centran en equilibrar 3 aspectos: escalabilidad, disponibilidad, \textit{throughput} y rendimiento.
		\paragraph{\textit{Clusters}/WSC (\textit{Warehouse Scale Computers}).} 
		Al igual que apartado anterior, hace énfasis en la escalabilidad, disponibilidad, \textit{throughput} y rendimiento. Distribución conjunta de ordenadores LAN y servidores para ofrecer SaaS (\textit{Software as a Service}): búsquedas, vídeo compartido, videojuegos multijugador...
		\paragraph{Embebidos.} 
		Espectro muy amplio. Coche, microondas, cajeros y entre ejemplos.
		
		\subsection{Diseño de computadores.}
		En primer lugar, es necesario establecer requerimientos funcionales. Especificar el área de aplicación, compatibilidad, sistema operativo y estándares a emplear.
		
		\subsubsection{Decisión de implementación.}
		
		\paragraph{Por \textit{software}.} 
		Supone un bajo coste errores, además facilitar el diseño y la actualización de esta.
		\paragraph{Por \textit{hardware}.} 
		Ventaja en rendimiento.
		
		\subsubsection{Consideración de nuevas tendencias.}
		Dado que arquitectura o tecnología puede permanecer en el mercado por mucho tiempo.
		
		\paragraph{Tendencias \textit{hardware}.} Nuevas tecnologías de almacenamiento secundario como M.2 o SSD, en cuanto a la memoria primaria como DRR5, aumento del número de transistores.
		\paragraph{Ley de Moore.} Parámetros como las velocidades de cómputo se duplican cada 18 meses. En la actualidad, no se cumple.
		\paragraph{Tendencias \textit{software}.} Programas requieren mayor memoria, nuevos lenguajes de programación.
		
		\subsection{Principios de diseño.}
		
		\subsubsection{Ley de Amdahl.} 
		Busca favorecer el caso frecuente.
		\begin{displaymath}
			a_{rendimiento} = \frac {R_{mejorado}} {R_{original}} =  \frac {t_{original}} {t_{mejorado}} 
		\end{displaymath}
		
		\subsubsection{Ley de rendimientos decrecientes.}
		La aceleración decrece tanto como se vayan añadiendo mejoras.
		
		\subsubsection{Localidad de referencia.}
		Los programas suelen emplear el 90\% de su tiempo de ejecución en el 10\% del código.
		
		\paragraph{Localidad temporal.} Los elementos accedidos recientemente probablemente volverán a ser solicitados en el futuro.
		
		\paragraph{Localidad espacial.} Los elementos cuyas direcciones cercanas tienden a ser referenciados juntos en el tiempo.
		
		\section{Rendimiento.}
		\subsection{Evolución.}
		Los computadores han evolucionado muy rápido durante los últimos 70 años. Algunas de las claves son:
		
		\begin{itemize}
			\item \textbf{Avances en la tecnología.} Se reduce el tamaño los elementos en el chip (\textit{feature size}), clk...
			\item \textbf{Innovaciones en el diseño.} Compiladores de alto nivel, estándar UNIX y amplio uso de la arquitectura RISC (\textit{Reduced Instruction Set Computer}).
		\end{itemize}

		\paragraph{Finales de 1970.} Aparece el microprocesador. Tasa alta de mejora anual: 25-36\%. Producción masiva. Eliminación de la necesidad para programar ensamblador. Sistemas operativos estandarizados como UNIX, lo que reduce costes y riesgos de una nueva arquitectura.
		\paragraph{Mitad de los años 80.} Aparición de primeras máquinas RISC. Enfoque en la mejora de rendimiento por el uso de cachés y provecho del paralelismo a nivel de instrucción.
		\paragraph{Hasta inicios de los 2000.} \% mejora es constante.
		\paragraph{Inicios de los 2000.} El alcance de la máxima disipación de chips refrigerados por aire, y la falta de paralelismo a nivel de instrucción. Estos factores redujeron a la mitad el \% de mejora.
		
		\subsection{Concepto.}
		El rendimiento de computador son un conjunto de métricas destinadas a parametrizar el desempeño de un ordenador en una serie determinada de tareas y necesidades.
		
		\paragraph{}
		Puede que a un usuario le interese reducir el $t_{respuesta}$/$t_{ejecuci\acute{o}n}$ para ejecutar más programas en un menor tiempo. Mientras que para la administración de \textit{clusters} o servidores se centre en aumentar la relación $transaction$/$hora$.
		
		\paragraph{}
		Dado que en la métrica de rendimiento influyen factores no determinísticos, se emplea distribuciones de probabilidad.
		
		\begin{displaymath}
				Rendimiento = \frac 1 t
		\end{displaymath}
		
		\paragraph{}
		Decimos que \textit{X} es un \textit{n} \% más rápido que \textit{Y}. 
		
		\begin{displaymath}
				t_{x} + \frac n {100} t_{x} = t_{y} 
		\end{displaymath}		
		\begin{displaymath}
				A = \frac {t_{antiguo}} {t_{nuevo}} = \frac {t_{y}} {t_{x}} = 1 + \frac n {100}
		\end{displaymath}
		
		\paragraph{}
		Incremento anual del rendimiento ($R$) en base a un año ($n$) anterior ($n-1$).
		 
		\begin{displaymath}
			R_{n} = \Delta_{anual} \cdot R_{n-1} = (\Delta_{anual})^n \cdot R_0
		\end{displaymath}
		
		\begin{displaymath}
			\Delta_{anual} = \sqrt[n]{\frac {R_n} {R_0}} = \sqrt[n]{\frac {t_0} {t_n}} 
		\end{displaymath}
		
		\subsection{Aplicaciones de Amdahl.}
		Define la ganancia de rendimiento o aceleración que puede obtenerse al mejorar alguna característica de un computador. Para ello, hemos de distinguir entre:
		
		\paragraph{No mejorable.}
		Refiere a aquella parte en la mejora no puede ser aplicada.
		
		\paragraph{Mejorable.}
		Parte en la que una mejora puede ser aplicada.
		
		\paragraph{Mejorado.}
		Parte en la que una mejora es aplicada.
		
		\paragraph{Fracción mejorada.}
		Es la fracción de tiempo sobre la opción sin mejora que puede utilizarse para optimizar. 
		
		\subparagraph{Ejemplo:}
		Un programa dura 10 segundos de los cuales 4 realiza trabajo en coma flotante. Este último puede mejorarse. Entonces: \newline $F_{mejorada} = \frac {4} {10}$.
		
		\paragraph{Aceleración mejorada.}
		Aceleración solo en la parte que está mejorada o que sea mejorable. No debe confundirse con la aceleración global.
		
		\paragraph{Aceleración global.}
		Es el cálculo de aceleración aplicado al cómputo de ambas partes que no puede aplicarse.
		
		\subparagraph{SIguiendo con el ejemplo anterior:}
		La mejora a implementar supone un cambio significante. Pasando de tardar 4 segundos a 1 solo segundo. Entonces: \newline $A_{mejorada\ original} = \frac {4} {4} = 1$ \newline $A_{mejorada\ con\ mejora} = \frac {t_{antiguo}} {t_{nuevo}} = \frac {4} {1} = 4$.
		
		\paragraph{El tiempo de ejecución nuevo.}
		Se define como:
		\begin{displaymath}
			t_{nuevo} = t_{antiguo} \cdot ( 1 - F_{mejorada} + \frac {F_{mejorada}} {A_{mejorada}})
		\end{displaymath}
		
		$F_{mejorada} = 1 \implies A_{global} = A_{mejorada}$
		
		$F_{mejorada} = 0 \implies A_{global} = 1$
		
		\subsection{Relación de coste.}
		El coste es un parámetro a tener en cuenta. Está influenciado por estos factores:
		\paragraph{Curva de aprendizaje.}
		Conforme avanza el tiempo: 
		\begin{itemize}
			\item Los costes de manufacturación se van reduciendo pese a no haber mejora.
			\item Aumenta el número de dispositivos que superan las pruebas.
		\end{itemize}
		\paragraph{Volumen.}
		Al incrementar el volumen de producción:
		\begin{itemize}
			\item Menos tiempo es necesario para mejorar los procesos (o bajar la curva de aprendizaje).
			\item Eficiencia en la manufactura y compra.
			\item El coste de diseño se amortiza entre más productos.
		\end{itemize}
		\paragraph{Mercado altamente competitivo.}
		Existe una gran variedad de compañías que cuentan con diverso catálogo. Junto al volumen, se reduce la diferencia entre el precio de venta y el coste.
		
		\subsection{Coste de un circuito integrado (IC).}
		El coste de silicio depende del número de puertas, conexiones y la regularidad del diseño (más es mejor, ocupará menos área).
		\paragraph{}
		Están definidos por el \textit{feature size}. Esto se refiere al tamaño que pueden tener los componentes dentro de un circuito integrado.
		\paragraph{}
		El complejo proceso de fabricación del silicio no ha variado considerablemente: la oblea es probada y cortada en dados que son empaquetados. Los defectos se desechan.
	
		\begin{displaymath}
			C_{IC}=\frac {C_{dado} + C_{test\ dado} + C_{empaquetar+test\ final}} {N_{final\ yield}}
		\end{displaymath}
		\begin{displaymath}
			C_{dado}=\frac {C_{oblea}} {N_{dados/oblea} \cdot R_{die\ yield}}
		\end{displaymath}
		\begin{displaymath}
			N_{dados/oblea} = \frac {\pi ({d_{oblea}} / 2)^2} {A_{oblea}} - \frac {\pi {d_{oblea}}} {\sqrt {2 A_{dado}}}
		\end{displaymath}
		\begin{displaymath}
			R_{die\ yield} = R_{wafer\ yield} \cdot \frac 1 {1+ (N_{defectos\ \acute{a}rea} \cdot A_{dado})^M}
		\end{displaymath}
		
		En donde,
		
		\begin{tabular}{ll}
			$C$: & Coste. \\
			$N$: & Número. Cantidad. Cardinalidad. \\
			$R$: & Rendimiento. Expresado en fracción.\\
			$d$: & Diámetro. \\
			$A$: & Área. Superficie. \\
			$M$: & Complejidad del proceso de fabricación. \\
			yield: & Sin defectos. \\
			oblea: & Wafer. Lámina de silicio. \\
			dado: & Die. Tiene impreso el circuito. \\
		\end{tabular}
		
		\subsection{Métricas.}
		
		\paragraph{Tiempo de reloj. Tiempo de respuesta.}
		Tiempo transcurrido para completar una tarea incluyendo todo.
		
		\paragraph{Ciclos por instrucción (CPI).}
		Número medio de ciclos reloj por instrucción.
		
		\begin{displaymath}
			CPI = \frac {N_{ciclos}} {RI}
		\end{displaymath}
		
		En donde $RI$ es recuento de instrucción.
		
		\paragraph{Tiempo de CPU.}
		Tiempo en que la CPU realiza tareas sin incluir tiempos de espera para
		E/S u otros programas.
		
		\begin{displaymath}
			t_{CPU} = N_{ciclos} \cdot t_{reloj/ciclo} = \frac {N_{ciclos}} {f_{reloj}}
		\end{displaymath}
		\begin{displaymath}
			t_{CPU} =  N_{ciclos} \cdot clk = RI \cdot CPI \cdot clk
		\end{displaymath}
		
		Para $clk$ es la duración de reloj y $RI$ es el recuento de instrucciones.
		
		\paragraph{Tiempo de CPU y CPI en detalle.}
		Se puede detallar el CPI por cada $i$ tipo de instrucción.
		
		\begin{displaymath}
			N_{ciclos} = \sum_{i=1}^{n} {CPI_i \cdot RI_i}
		\end{displaymath}
		
		Lo que permite expresar:
		
		\begin{displaymath}
			t_{CPU} = clk \cdot \sum_{i=1}^{n} {CPI_i \cdot RI_i}
		\end{displaymath}
		\begin{displaymath}
			CPI = \frac {\sum_{i=1}^{n} {CPI_i \cdot RI_i}} {RI}
		\end{displaymath}
		
		\paragraph{Millones de instrucciones por segundo (MIPS).}
		Más fiable, puesto que involucra el tiempo en ejecución de programas reales. Esta métrica no debería usarse para comparativas con distintos repertorios de instrucciones porque los $MIPS$ pueden variar inversamente al rendimiento.
		\begin{displaymath}
			MIPS = \frac {RI} {t_{ejecuci\acute{o}n} \cdot 10^6} = \frac {f_{reloj}} {CPI \cdot 10^6}
		\end{displaymath}
		
		\paragraph{MIPS de referencia.}
		Usado para comparar usando otra máquina como referencia. Pues se apoya en el tiempo en ejecución de un determinado programa.
		
		\begin{displaymath}
			MIPS_{relativos} = \frac {t_{referencia}} {t_{no\ estimado}} MIPS_{referencia}
		\end{displaymath}
		
		Sea ${t_{referencia}}$ el tiempo transcurrido de un programa en la máquina de referencia, y
		${t_{no\ estimado}}$ = tiempo de la aplicación en el dispositivo que se va a medir.
		
		\paragraph{Operaciones de punto flotante por segundo (FLOPS).}
		Dependen del repertorio de instrucciones. La comparativa puede ser inconsistente.
		
		\begin{displaymath}
			MFLOPS = \frac {N_{operaciones\ flotante}} {t_{ejeci\acute{o}n} \cdot 10^6}
		\end{displaymath}
		
		\paragraph{FLOPS Normalizados.}
		Livermore Loops propone solucionar el problema de los FLOPS añadiendo "\textit{pesos}" a diferentes operaciones.
		
		\subsection{\textit{Benchmarks}.}
		Serie de pruebas para determinar el $t_{ejecuci\acute{o}n}$ (u otro tipo de puntuación) sobre la carga de trabajo del usuario (\textit{workload}).
		
		\paragraph{Aplicaciones reales.}
		Se hacen pruebas usando: compiladores (g++, gcc, clang...), procesamiento de texto (TeX, Jupyter Notebook...) y herramientas CAD (blender, Spice...).
		
		\paragraph{Núcleos.}
		Son pequeños fragmentos clave de programas. Livermore Loops y Linpack.
		
		\paragraph{\textit{Benchmarks} reducidos (\textit{Toys}).}
		Pruebas específicas reducidas portables, de fácil simulación y estandarizadas: multiplicación de matrices, \textit{quicksort}, obtención de números primos...
		
		\paragraph{\textit{Benchmarks} sintéticos).}
		Intención de simular la frecuencia media de operaciones y operandos de un gran conjunto de programas. Whetstone y Dhrystone.
		
		\paragraph{Colecciones de \textit{benchmarks}.}
		 Conjunto de pruebas que desafían al dispositivo bajo una gran variedad de escenarios. Formadas por núcleos y programas reales. 
		 \paragraph{}
		 Son pruebas reproducibles debido al detalle para la ejecución de estas (parámetros de compilación, números y tipos de disco, sistema operativo...)
		 \paragraph{}
		 Algunas de las colecciones más importantes son SPEC, Business Winstone, Winbench. 
		
	\end{multicols}
	
\end{document}
